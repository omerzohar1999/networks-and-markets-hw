\documentclass{article}
\usepackage[a4paper, margin=2.5cm]{geometry}
\usepackage{graphicx}
\usepackage{pgf}
\usepackage{pgfkeys}
\usepackage{caption}
\usepackage{amsfonts}
\usepackage{xcolor}
\usepackage{amsmath}
\usepackage{amsthm}
\usepackage{hyperref}
\usepackage{cleveref}
\usepackage{enumitem}
\usepackage{float}
\usepackage{diagbox}
\usepackage{tikz}
\usepackage{adjustbox}
\usepackage{pgfplots}
\usepackage{placeins} % To use \FloatBarrier
\usetikzlibrary{arrows.meta, positioning}

\hypersetup{
    colorlinks=true,
    linkcolor=blue,
    filecolor=magenta,
    urlcolor=cyan,
    citecolor=blue
}

\newtheorem{theorem}{Theorem}[section]  % Theorems numbered within sections
\newtheorem{lemma}[theorem]{Lemma}      % Lemma numbering follows theorem

% Define argmax
\DeclareMathOperator*{\argmax}{arg\,max}  % The asterisk is used to allow limits to go underneath in displaystyle

% Apply color to cref and Cref
\crefformat{figure}{#2figure\color{blue}~#1#3}
\Crefformat{figure}{#2Figure\color{blue}~#1#3}
\crefformat{section}{#2section\color{blue}~#1#3}
\Crefformat{section}{#2Section\color{blue}~#1#3}
\crefformat{table}{#2table\color{blue}~#1#3}
\Crefformat{table}{#2Table\color{blue}~#1#3}

% color definitions
\definecolor{darkgreen}{rgb}{0.0, 0.7, 0.0} % A darker green

% Define the keys used in the graph
\pgfkeys{
    /graph/.is family, /graph,
    default/.style={
        buyers={}, 
        items={}, 
        buyerValues={}, 
        prices={},
        priceVectorName=p % Default name
    },
    buyers/.estore in=\GraphBuyers,
    items/.estore in=\GraphItems,
    buyerValues/.estore in=\GraphBuyerValues,
    prices/.estore in=\GraphPrices,
    priceVectorName/.estore in=\GraphPriceVectorName % Store the custom price vector name
}

% Helper macro to extract prices by index
\newcommand{\getPrice}[1]{
    \foreach \price [count=\i] in \GraphPrices {
        \ifnum\i=#1
            \xdef\selectedPrice{\price}
            \breakforeach % Stop the loop once the correct item is found
        \fi
    }
}

% Helper macro to extract values by index
\newcommand{\getValue}[2]{
    \foreach \value [count=\i] in \GraphBuyerValues{
        \ifnum\i=#1
            \foreach \val [count=\j] in \value {
                \ifnum\j=#2
                    \xdef\selectedValue{\val}
                    \breakforeach % Stop the loop once the correct item is found
                \fi
            }
        \fi
    }
}

% Helper macro to extract utilities by index
\newcommand{\getUtility}[2]{
    \getPrice{#2}  % Retrieve the price by index
    \getValue{#1}{#2}  % Retrieve the values by index
    \pgfmathsetmacro\utility{int(\selectedValue - \selectedPrice)} % Ensure result is an integer
}

% Helper macro to extract the indices of the maximum non-negative utilities for a buyer
\newcommand{\getPreferredUtilityIndices}[1]{
    \global\def\maxUtility{0} % Initialize maxUtility globally
    \def\preferredItems{-1} % List to store all best items
    \foreach \item [count=\ii] in \GraphItems {
        \typeout{currindices={\bi, \ii}}
        % Compute utility
        \getUtility{#1}{\ii} % Retrieve the utility by indices
        \typeout{utility={\utility}}

        % Check if this utility is the maximum for this buyer
        \ifnum\utility>\maxUtility
            \xdef\maxUtility{\utility}
            \xdef\preferredItems{\ii} % Reset with the new best item
        \else
            \ifnum\utility=\maxUtility
                \xdef\preferredItems{\preferredItems, \ii} % Append this item to the list
            \fi
        \fi
    }
    \typeout{Max Utility for \buyer: \maxUtility, Best Items: \preferredItems}
}

% Helper macro to extract the indices of non-negative utilities for a buyer
\newcommand{\getAcceptableUtilityIndices}[1]{
    \def\acceptableItems{-1} % List to store all acceptable items
    \foreach \item [count=\ii] in \GraphItems {
        \typeout{currindices={\bi, \ii}}
        % Compute utility
        \getUtility{#1}{\ii} % Retrieve the utility by indices
        \typeout{utility={\utility}}

        % Check if this utility is non-negative
        \ifnum\utility<0
        \else
            \xdef\acceptableItems{\acceptableItems, \ii} % Append this item to the list
        \fi
    }
    \typeout{Acceptable Items for \buyer: \acceptableItems}
}

% New macro to construct "\item : \value" list
\newcommand{\getValueDict}[1]{
    \def\valueList{}% Global definition to ensure it is accessible outside the scope of this macro
    \foreach \item [count=\ii] in \GraphItems {
        \getValue{#1}{\ii}% This should set \selectedValue appropriately
        \typeout{value={\selectedValue}}
        \ifnum\ii=1
            \xdef\valueList{\item: \selectedValue}
        \else
            \xdef\valueList{\valueList, \item: \selectedValue}
        \fi
    }
}

% New macro to construct "\item : \utility" acceptability list
\newcommand{\getAcceptableUtilityDict}[1]{
    \def\utilityList{}% Global definition to ensure it is accessible outside the scope of this macro
    \foreach \item [count=\ii] in \GraphItems {
        \getUtility{#1}{\ii}% This should set \utility appropriately
        \xdef\utilityString{\item: \utility} % Use \edef for expanded definition
        \ifnum\utility < 0
        \else
            \xdef\utilityString{\item: \textcolor{darkgreen}{\utility}} % Color non-negative utilities
        \fi
        \ifnum\ii=1
            \xdef\utilityList{\utilityString} % Initialize utilityList with the first item
        \else
            \xdef\utilityList{\utilityList, \utilityString} % Append subsequent items
        \fi
    }
}

% New macro to get max non-negative utility
\newcommand{\getMaxUtility}[1]{
    \def\maxUtility{-1} % Initialize maxUtility
    \foreach \item [count=\ii] in \GraphItems {
        \getUtility{#1}{\ii} % Retrieve the utility by index
        \ifnum\utility<0
        \else
            \ifnum\utility>\maxUtility
                \xdef\maxUtility{\utility}
            \fi
        \fi
    }
}

% New macro to construct "\item : \utility" preferred list
\newcommand{\getPreferredUtilityDict}[1]{
    \def\utilityList{}% Global definition to ensure it is accessible outside the scope of this macro
    \getMaxUtility{#1} % Retrieve the maximum non-negative utility
    \foreach \item [count=\ii] in \GraphItems {
        \getUtility{#1}{\ii}% This should set \utility appropriately
        \xdef\utilityString{\item: \utility} % Use \edef for expanded definition
        \ifnum\utility < 0 % only non-negative utilities can be preferred
        \else
            \ifnum\utility = \maxUtility

                \xdef\utilityString{\item: \textcolor{darkgreen}{\utility}} % Color non-negative utilities
            \fi
        \fi
        \ifnum\ii=1
            \xdef\utilityList{\utilityString} % Initialize utilityList with the first item
        \else
            \xdef\utilityList{\utilityList, \utilityString} % Append subsequent items
        \fi
    }
}

% Define the command to create the graph
\newcommand{\createPreferredGraph}[1]{
    \pgfkeys{/graph/.cd,#1}  % Apply settings within the /graph family
    \expandafter\parsePreferredGraphData\expandafter{\GraphBuyers}{\GraphItems}{\GraphBuyerValues}{\GraphPrices}{\GraphPriceVectorName}
}

% Helper macro to process the graph data and render the tikzpicture
\newcommand{\parsePreferredGraphData}[5]{
    \begin{figure}[H]
        \centering
        \begin{tikzpicture}[every node/.style={align=center}]
            % Define nodes for buyers
            \typeout{msg={#1}}
            \foreach \buyer [count=\bi] in {#1} {
                \node[draw, circle] (buyer-\bi) at (0, {-2*\bi}) {\buyer};
            }
            % Define nodes for items
            \typeout{msg={#2}}
            \typeout{msg={#4}}
            \foreach \item [count=\ii] in #2 {
                \typeout{msg={\ii}}
                \getPrice{\ii}  % Retrieve the price by index
                \node[draw, circle] (item-\ii) at (4, {-2*\ii}) {\item \\ $#5_{\item} = \selectedPrice$};
            }
            % Calculate utilities and draw preferred choices
            \foreach \buyer [count=\bi] in {#1} {

                % Get indices of preferred items
                \getPreferredUtilityIndices{\bi}

                % Draw edges to preferred items
                \typeout{preferred={Preferred Items: \preferredItems}}
                \foreach \ii in \preferredItems {
                    \ifnum\ii > 0
                        \draw (buyer-\bi) -- (item-\ii);
                    \fi
                }

                % List values and utilities
                \getValueDict{\bi}
                \getPreferredUtilityDict{\bi}

                \node[anchor=east] at (-1, {-2*\bi + 0.3}) {Values for $\buyer$: $\{\valueList\}$};
                \node[anchor=east] at (-1, {-2*\bi - 0.3}) {Utility for $\buyer$: $\{\utilityList\}$};
            }
        \end{tikzpicture}
    \end{figure}
}

% Define the command to create the graph
\newcommand{\createAcceptabilityGraph}[1]{
    \pgfkeys{/graph/.cd,#1}  % Apply settings within the /graph family
    \expandafter\parseAcceptabilityGraphData\expandafter{\GraphBuyers}{\GraphItems}{\GraphBuyerValues}{\GraphPrices}{\GraphPriceVectorName}
}

% Helper macro to process the graph data and render the tikzpicture
\newcommand{\parseAcceptabilityGraphData}[5]{
    \begin{figure}[H]
        \centering
        \begin{tikzpicture}[every node/.style={align=center}]
            % Define nodes for buyers
            \typeout{msg={#1}}
            \foreach \buyer [count=\bi] in {#1} {
                \node[draw, circle] (buyer-\bi) at (0, {-2*\bi}) {\buyer};
            }
            % Define nodes for items
            \typeout{msg={#2}}
            \typeout{msg={#4}}
            \foreach \item [count=\ii] in #2 {
                \typeout{msg={\ii}}
                \getPrice{\ii}  % Retrieve the price by index
                \node[draw, circle] (item-\ii) at (4, {-2*\ii}) {\item \\ $#5_{\item} = \selectedPrice$};
            }
            % Calculate utilities and draw acceptable choices
            \foreach \buyer [count=\bi] in {#1} {

                % Get indices of acceptable items
                \getAcceptableUtilityIndices{\bi}

                % Draw edges to acceptable items
                \typeout{acceptable={Acceptable Items: \acceptableItems}}
                \foreach \ii in \acceptableItems {
                    \ifnum\ii > 0
                        \draw (buyer-\bi) -- (item-\ii);
                    \fi
                }

                % List values and utilities
                \getValueDict{\bi}
                \getAcceptableUtilityDict{\bi}

                \node[anchor=east] at (-1, {-2*\bi + 0.3}) {Values for $\buyer$: $\{\valueList\}$};
                \node[anchor=east] at (-1, {-2*\bi - 0.3}) {Utility for $\buyer$: $\{\utilityList\}$};
            }
        \end{tikzpicture}
    \end{figure}
}

% Define the command to create a matching market graph
\newcommand{\createMatchingMarketGraph}[1]{
    \pgfkeys{/graph/.cd,#1}  % Apply settings within the /graph family
    \expandafter\parseMatchingMarketGraphData\expandafter{\GraphBuyers}{\GraphItems}{\GraphBuyerValues}
}

% Helper macro to process the graph data and render the tikzpicture
\newcommand{\parseMatchingMarketGraphData}[3]{
    \begin{figure}[H]
        \centering
        \begin{tikzpicture}[every node/.style={align=center}]
            % Define nodes for buyers
            \typeout{msg={#1}}
            \foreach \buyer [count=\bi] in {#1} {
                \node[draw, circle] (buyer-\bi) at (0, {-2*\bi}) {\buyer};
            }
            % Define nodes for items
            \typeout{msg={#2}}
            \foreach \item [count=\ii] in #2 {
                \node[draw, circle] (item-\ii) at (4, {-2*\ii}) {\item};
            }
            % Calculate utilities and draw acceptable choices
            \foreach \buyer [count=\bi] in {#1} {
                \getValueDict{\bi}
                \node[anchor=east] at (-1, {-2*\bi + 0.3}) {Values for \buyer: $\{\valueList\}$};
            }
        \end{tikzpicture}
    \end{figure}
}

% start
\title{
    Homework Assignment 3 - Coding Part Write-up\\
    Networks and Markets
}
\author{
    Omer Zohar
    \and
    Gil Aharoni
    \and
    Adam Tuby
}

\bibliographystyle{plain}

\begin{document}
\maketitle

\section*{Part 4: Implementing Matching Market Pricing}
\setcounter{section}{0}


\section{Question 7}

\begin{enumerate}[label=(\alph*)]

    \item[(b)] Consider the matching market example in Lecture 5 Page 7:
    
    \createMatchingMarketGraph{
        buyers={Person A, Person B, Person C},
        items={House 1, House 2, House 3},
        buyerValues={{4, 12, 5}, {7, 10, 9}, {7, 7, 10}}
    }

    Formally, the matching market context is $\Gamma = (\{A, B, C\}, \{1, 2, 3\}, v)$, where $v$ is the valuation function defined as follows:

    \begin{align*}
        v_A(1) &= 4, v_A(2) = 12, v_A(3) = 5 \\
        v_B(1) &= 7, v_B(2) = 10, v_B(3) = 9 \\
        v_C(1) &= 7, v_C(2) = 7, v_C(3) = 10
    \end{align*}

    We turn to run the algorithm of Theorem 8.8 to find a market equilibrium $(p, M)$ to find the maximum social value, in order to validate out implementation's output. We begin by initializing the prices vector $\vec{p} \equiv 0$ to be the zero vector. We then proceed to run the algorithm, updating the prices vector until there is a perfect matching $M$ in the induced preferred choice graph for $(\Gamma, \vec{p})$:

    \begin{enumerate}[label=\arabic*.]

        \item Observing the following \textit{induced preferred-choice graph} from $(\Gamma, \vec{p})$:
        
        \createPreferredGraph{
            buyers={A, B, C},
            items={1, 2, 3},
            buyerValues={{4, 12, 5}, {7, 10, 9}, {7, 7, 10}},
            prices={0, 0, 0},
            priceVectorName=p
        }

        There obviously isn't a perfect matching as $S = \{A, B\}$ is a constricted set with $\left|N(S)\right| = \left|\{2\}\right| = 1 < 2 = \left|S\right|$ (which, by a theorem we've seen in class, implies that there isn't a perfect matching). Thus, we raise the prices for all items in $N(S)$ by 1, and update the prices vector $\vec{p}$ accordingly. The updated prices vector is $\vec{p} = (a: 0, b: 1, c: 0)$. Not all prices are greater than zero, so we don't perform a shift operation, and we proceed to the next iteration.

        \item Observing the following \textit{induced preferred-choice graph} from $(\Gamma, \vec{p})$:
        
        \createPreferredGraph{
            buyers={A, B, C},
            items={1, 2, 3},
            buyerValues={{4, 12, 5}, {7, 10, 9}, {7, 7, 10}},
            prices={0, 1, 0},
            priceVectorName=p
        }

        There obviously isn't a perfect matching as $S = \{A, B, C\}$ is a constricted set with $\left|N(S)\right| = \left|\{2, 3\}\right| = 2 < 3 = \left|S\right|$ (which, by a theorem we've seen in class, implies that there isn't a perfect matching). Thus, we raise the prices for all items in $N(S)$ by 1, and update the prices vector $\vec{p}$ accordingly. The updated prices vector is $\vec{p} = (a: 0, b: 2, c: 1)$. Not all prices are greater than zero, so we don't perform a shift operation, and we proceed to the next iteration.

        \item Observing the following \textit{induced preferred-choice graph} from $(\Gamma, \vec{p})$:
        
        \createPreferredGraph{
            buyers={A, B, C},
            items={1, 2, 3},
            buyerValues={{4, 12, 5}, {7, 10, 9}, {7, 7, 10}},
            prices={0, 2, 1},
            priceVectorName=p
        }

        Similar to the previous iteration, we raise the prices for $\{2, 3\}$, and update the prices vector $\vec{p}$ accordingly. The updated prices vector is $\vec{p} = (a: 0, b: 3, c: 2)$. Not all prices are greater than zero, so we don't perform a shift operation, and we proceed to the next iteration.

        \item Observing the following \textit{induced preferred-choice graph} from $(\Gamma, \vec{p})$:
        
        \createPreferredGraph{
            buyers={A, B, C},
            items={1, 2, 3},
            buyerValues={{4, 12, 5}, {7, 10, 9}, {7, 7, 10}},
            prices={0, 3, 2},
            priceVectorName=p
        }

        And there is a perfect matching in the induced preferred choice graph, which is $M = \{\{A, 2\}, \{B, 1\}, \{C, 3\}\}$. Thus, the market equilibrium is $(\vec{p}, M) = ((1: 0, 2: 3, 3: 2), \{\{A, 2\}, \{B, 1\}, \{C, 3\}\})$, and we are done.
    \end{enumerate}

    We found the market equilibrium to be $(\vec{p}, M) = ((1: 0, 2: 3, 3: 2), \{\{A, 2\}, \{B, 1\}, \{C, 3\}\})$. The maximum social value is therefore $v(A, 2) + v(B, 1) + v(C, 3) = 12 + 7 + 10 = 29$.

    Our algorithm found exactly this market equilibrium.

\end{enumerate}

\section{Question 8}

\begin{enumerate}[label=(\alph*)]

    \item In this part we analyze how the prices output by the VCG mechanism compare with the ones output by the algorithm of Theorem 8.8 (finding a market equilibrium $(p, M)$). The following are the examples we analyze and their corresponding results for each mechanism:
    
    \begin{enumerate}[label=\arabic*.]
        \item Example 1:

        % [q8b_analysis] test 0:
        % [q8b_analysis][graph]: n=7 m=3 test_V=array([[ 3,  0, 15],
        %        [11, 12,  8],
        %        [14, 13, 18],
        %        [ 3, 18,  5],
        %        [15,  9,  1],
        %        [14, 17,  9],
        %        [14, 17, 13]])
        % [q8b_analysis][market_eq]: P_eq=[14, 17, 15], M_eq=[None, None, 2, 1, 0, None, None]
        % [q8b_analysis][vcg]: P_vcg=[14.0, 17.0, 15.0], M_vcg=[None, None, 2, 1, 0, None, None]
        % [q8b_analysis][different?]: False

        \FloatBarrier

        \createMatchingMarketGraph{
            buyers={$b_1$, $b_2$, $b_3$, $b_4$, $b_5$, $b_6$, $b_7$},
            items={1, 2, 3},
            buyerValues={{3, 0, 15}, {11, 12, 8}, {14, 13, 18}, {3, 18, 5}, {15, 9, 1}, {14, 17, 9}, {14, 17, 13}}
        }

        \FloatBarrier

        And we observe that the prices output by the VCG mechanism and the algorithm of Theorem 8.8 are the same (the matching is also the same because we used the same algorithm to compute the socially optimal state as part of the VCG mechanism)

        \item Example 2:

        % [q8b_analysis] test 1:
        % [q8b_analysis][graph]: n=6 m=6 test_V=array([[12, 14, 16,  8,  6, 17],
        %        [11,  7,  9, 19,  1, 11],
        %        [18, 13, 17, 17,  2, 16],
        %        [15,  0,  4,  1, 15, 15],
        %        [ 7,  8,  5, 12, 18, 13],
        %        [ 7, 19,  8, 12,  4,  1]])
        % [q8b_analysis][market_eq]: P_eq=[1, 0, 0, 0, 1, 1], M_eq=[2, 3, 0, 5, 4, 1]
        % [q8b_analysis][vcg]: P_vcg=[1.0, 0.0, 0.0, 0.0, 1.0, 1.0], M_vcg=[2, 3, 0, 5, 4, 1]
        % [q8b_analysis][different?]: False

        \FloatBarrier

        \createMatchingMarketGraph{
            buyers={$b_1$, $b_2$, $b_3$, $b_4$, $b_5$, $b_6$},
            items={1, 2, 3, 4, 5, 6},
            buyerValues={{12, 14, 16, 8, 6, 17}, {11, 7, 9, 19, 1, 11}, {18, 13, 17, 17, 2, 16}, {15, 0, 4, 1, 15, 15}, {7, 8, 5, 12, 18, 13}, {7, 19, 8, 12, 4, 1}}
        }

        \FloatBarrier

        And we observe that the prices output by the VCG mechanism and the algorithm of Theorem 8.8 are the same (the matching is also the same because we used the same algorithm to compute the socially optimal state as part of the VCG mechanism)

        \item Example 3:

        % [q8b_analysis] test 2:
        % [q8b_analysis][graph]: n=6 m=6 test_V=array([[ 8, 11,  0,  3,  6,  7],
        %        [19, 14, 15, 14, 14, 16],
        %        [17, 19, 19, 13,  8, 17],
        %        [ 2, 15,  1, 18, 11, 10],
        %        [ 8,  9,  7, 15,  6, 10],
        %        [12, 15, 15,  8,  2,  1]])
        % [q8b_analysis][market_eq]: P_eq=[2, 4, 4, 7, 0, 2], M_eq=[1, 0, 5, 4, 3, 2]
        % [q8b_analysis][vcg]: P_vcg=[2.0, 4.0, 4.0, 7.0, 0.0, 2.0], M_vcg=[1, 0, 5, 4, 3, 2]
        % [q8b_analysis][different?]: False

        \FloatBarrier

        \createMatchingMarketGraph{
            buyers={$b_1$, $b_2$, $b_3$, $b_4$, $b_5$, $b_6$},
            items={1, 2, 3, 4, 5, 6},
            buyerValues={{8, 11, 0, 3, 6, 7}, {19, 14, 15, 14, 14, 16}, {17, 19, 19, 13, 8, 17}, {2, 15, 1, 18, 11, 10}, {8, 9, 7, 15, 6, 10}, {12, 15, 15, 8, 2, 1}}
        }

        \FloatBarrier

        And we observe that the prices output by the VCG mechanism and the algorithm of Theorem 8.8 are the same (the matching is also the same because we used the same algorithm to compute the socially optimal state as part of the VCG mechanism)

        \item Example 4:

        % [q8b_analysis] test 3:
        % [q8b_analysis][graph]: n=6 m=10 test_V=array([[ 5,  3,  0,  7, 10,  5, 17,  6, 18,  8],
        %        [ 5,  4,  6,  9, 15,  9, 17,  2, 10, 14],
        %        [10, 11, 10,  6,  4, 10, 16, 11, 10,  6],
        %        [ 2, 19,  4, 12,  5,  8, 12,  0, 11, 11],
        %        [18,  7, 15, 11,  7,  4,  2,  9,  9,  8],
        %        [ 5,  2,  2,  5,  1, 12, 13, 18,  8,  1]])
        % [q8b_analysis][market_eq]: P_eq=[0, 0, 0, 0, 0, 0, 2, 0, 0, 0], M_eq=[8, 4, 6, 1, 0, 7]
        % [q8b_analysis][vcg]: P_vcg=[0.0, 0.0, 0.0, 0.0, 0.0, 0.0, 2.0, 0.0, 0.0, 0.0], M_vcg=[8, 4, 6, 1, 0, 7]
        % [q8b_analysis][different?]: False    
        
        \FloatBarrier

        \createMatchingMarketGraph{
            buyers={$b_1$, $b_2$, $b_3$, $b_4$, $b_5$, $b_6$},
            items={1, 2, 3, 4, 5, 6, 7, 8, 9, 10},
            buyerValues={{5, 3, 0, 7, 10, 5, 17, 6, 18, 8}, {5, 4, 6, 9, 15, 9, 17, 2, 10, 14}, {10, 11, 10, 6, 4, 10, 16, 11, 10, 6}, {2, 19, 4, 12, 5, 8, 12, 0, 11, 11}, {18, 7, 15, 11, 7, 4, 2, 9, 9, 8}, {5, 2, 2, 5, 1, 12, 13, 18, 8, 1}}
        }

        \FloatBarrier

        And we observe that the prices output by the VCG mechanism and the algorithm of Theorem 8.8 are the same (the matching is also the same because we used the same algorithm to compute the socially optimal state as part of the VCG mechanism)

        \item Example 5:

        % [q8b_analysis][graph]: n=6 m=4 test_V=array([[15,  3,  1,  5],
        % [11, 11, 16,  5],
        % [ 9, 15, 13, 17],
        % [15, 11, 10, 16],
        % [19,  0, 12,  7],
        % [17, 16, 13,  9]])
        % [q8b_analysis][market_eq]: P_eq=[15, 14, 12, 16], M_eq=[None, 2, 3, None, 0, 1]
        % [q8b_analysis][vcg]: P_vcg=[15.0, 14.0, 12.0, 16.0], M_vcg=[None, 2, 3, None, 0, 1]
        % [q8b_analysis][different?]: False

        \FloatBarrier

        \createMatchingMarketGraph{
            buyers={$b_1$, $b_2$, $b_3$, $b_4$, $b_5$, $b_6$},
            items={1, 2, 3, 4},
            buyerValues={{15, 3, 1, 5}, {11, 11, 16, 5}, {9, 15, 13, 17}, {15, 11, 10, 16}, {19, 0, 12, 7}, {17, 16, 13, 9}}
        }

        \FloatBarrier

        And we observe that the prices output by the VCG mechanism and the algorithm of Theorem 8.8 are the same (the matching is also the same because we used the same algorithm to compute the socially optimal state as part of the VCG mechanism)
    \end{enumerate}
    
    That is, in all examples we analyzed, the prices output by the VCG mechanism and the algorithm of Theorem 8.8 were the same, and the matching was also the same because we used the same algorithm to compute the socially optimal state as part of the VCG mechanism. We analyzed far more examples besides the ones presented here, and the results were consistent across all of them---the prices output by the VCG mechanism and the algorithm of Theorem 8.8 were the same (and the matching was also the same because we used the same algorithm to compute the socially optimal state as part of the VCG mechanism).

\end{enumerate}

\section{Bonus Question 2}

\begin{enumerate}[label=(\alph*)]

    \item 

    
    \item 


    \item 

    
    \item 


\end{enumerate}

\section*{Part 5: Exchange Networks for Uber}
\setcounter{section}{0}


\section{Question 9}

\section{Question 10}

\begin{enumerate}[label=(\alph*)]

    \item 

    
    \item 


\end{enumerate}


\section{Question 11}

\section{Bonus Question 3}

\begin{enumerate}[label=(\alph*)]

    \item 

    
    \item 

\end{enumerate}

\bibliography{bibliography}

\end{document}